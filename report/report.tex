\documentclass[a4paper,english,12pt]{article}
%\usepackage[lwarpmk]{lwarp}
\usepackage{titlesec}
\usepackage{cc}
\usepackage{amsmath}
\usepackage[T1]{fontenc}
\usepackage[usenames,dvipsnames]{color}
\graphicspath{ {./images/} }
\usepackage[dvipsnames]{xcolor}
\titleformat*{\section}{\color{Brown}\normalfont\bfseries\Large}
\usepackage{prettyref}
\usepackage{amsthm}
\usepackage{amsmath}
\usepackage{amssymb}
\usepackage{float}
\usepackage{natbib}
\bibliographystyle{unsrtnat}
\PassOptionsToPackage{normalem}{ulem}
\usepackage{ulem}
%\usepackage{mathptmx}
\usepackage{framed}
\usepackage{array}
\usepackage{csvsimple,longtable,booktabs}
\newlength{\mycolwidth}
\settowidth{\mycolwidth}{2cm} % widest entry

%\usepackage[unicode=true,pdfusetitle,
% bookmarks=true,bookmarksnumbered=false,bookmarksopen=false,
% breaklinks=false,pdfborder={0 0 1},backref=section,colorlinks=false]{hyperref}

\usepackage{pdfcomment}

%\newcommand{\dania}[1]{}
%\newcommand{\joyce}[1]{}
\newcommand{\chenc}[1]{\pdfcomment[author=Chen,color={1 0.5 0.5},subject={#1}]{#1}}
\newcommand{\dania}[1]{\pdfcomment[author=Dania,color={1 1 1},subject={#1}]{#1}}
\newcommand{\joyce}[1]{\pdfcomment[author=Joyce,color={1 0 1},subject={#1}]{#1}}
%\geometry{bindingoffset=2cm}


\usepackage{lipsum}

\hypersetup{colorlinks,
	linkcolor=red!95!black,%YellowOrange!85!black,
	citecolor=blue!85!black,
	pagecolor=blue!95!black,%
        urlcolor=magenta,filecolor=magenta,breaklinks,%
        dvips,bookmarks,bookmarksopen}

%\hypersetup{
%  linkcolor  = violet!\myshade!black,
%  citecolor  = YellowOrange!\myshade!black,
%  urlcolor   = Aquamarine!\myshade!black,
%  colorlinks = true
%}

\makeatletter
%\usepackage{pgfplots}
%\usepackage{tikz}
%\usepgfplotslibrary{external}
%\usetikzlibrary{external}
%\usepgflibrary{ plotmarks }
%\usetikzlibrary{arrows}
%\usetikzlibrary{plotmarks}
%\tikzexternalize[prefix = resource/]%, mode=list and make
%\tikzset{external/mode=graphics if exists}
%% the following forces redraw of all, if you reorder graphs
%tikzset{external/force remake}
\usepackage{microtype}
%
\usepackage{doi}
%\bibpunct{(}{)}{;}{a}{,}{,}

\renewcommand{\MakeUppercase}[1]{\color{green!50!black}\textsf{#1}}
%\usepackage{pdfsync}
\usepackage{amsfonts}
\usepackage{amscd}
\usepackage{bm}
\usepackage{euler}
\usepackage{url}
%\graphicspath{resource/}
%\graphicspath{./resource/}
\newcommand{\pde}{{\textsc{pde}}}
\newcommand{\res}{{\operatorname{Res}}}
\newcommand{\D}[2]{\frac{\partial #1}{\partial #2}}
\newcommand{\DD}[2]{\frac{\partial^2 #1}{\partial #2^2}}
\newcommand{\pp}{{}+}
\newcommand{\Ord}{\mathcal{O}}
\newcommand{\LL}{\mathcal{L}}
\newcommand{\ode}{{\textsc{ode}}}
\newcommand{\cL}{\mathcal{L}}
\newcommand{\lhs}{{\textsc{lhs}}}
\newcommand{\rhs}{{\textsc{rhs}}}
\newcommand{\dd}[2]{\frac{\partial #1}{\partial #2}}

\usepackage{listings}
\lstset{escapeinside={(*@}{@*)}}
\usepackage{inconsolata}
\usepackage{textcomp}
%% Actual colors from idlelib/config-highlight.def --> corrected to ``web-safe''
%% strings  = #00aa00 / 0,170,0      (a darker green)
%% builtins = #900090 / 144,0,144    (purple-ish)
%% keywords = #FF7700 / 255,119,0    (quite close to plain `orange')
%% Corrected to ``web-safe''
\definecolor{purple2}{RGB}{153,0,153} % there's actually no standard purple
\definecolor{green2}{RGB}{0,153,0} % a darker green

\lstdefinestyle{python-idle-code}{%
  language=Python,                   % the language
  showspaces=false,
  showtabs=false,
  breaklines=true,
  showstringspaces=false,
  breakatwhitespace=true,
  basicstyle=\normalsize\ttfamily,   % size of the fonts for the code
  % Color settings to match IDLE style
  keywordstyle=\color{orange},       % core keywords
  keywordstyle={[2]\color{purple2}}, % built-ins
  stringstyle=\color{green2},
  commentstyle=\color{red},
  upquote=true,                      % requires textcomp
}
%\lstloadlanguages{Python}
\lstset{
  style={python-idle-code},
  showstringspaces=false,  % true by default for python
  % tabsize=4,
}
\lstset{
float=table,
stringstyle=\color{orange},
basicstyle=\color{black}\footnotesize\ttfamily,
numbers=left,
numberstyle=\tiny\color{brown},
%numberstyle=\small,
numbersep=8pt,
%frame = leftline,
breaklines=true,
firstnumber=1,
language=python,
numberstyle=\tiny\color{brown},
keywordstyle=\color{blue},
commentstyle=\color{green!50!black}}

% These are recommended by Rob J Hyndman (2011)
% \footnote{\url{http://robjhyndman.com/researchtips/latex-floats/}}
\setcounter{topnumber}{2}
\setcounter{bottomnumber}{2}
\setcounter{totalnumber}{4}
\renewcommand{\topfraction}{0.85}
\renewcommand{\bottomfraction}{0.85}
\renewcommand{\textfraction}{0.15}
\renewcommand{\floatpagefraction}{0.7}
\renewcommand{\lstlistingname}{Algorithm}
\usepackage[UKenglish]{babel}
%\includeonly{chapter3/chapter3}
\title{\normalfont \large  %14pt uncomment this line
Multi-class Classification using Multi-layer Neural Network}
\author{
    Chen Chen\quad
    Mingxuan Li\\
    480458339\quad
    470325230
}
\date{\normalfont\small \today}
\makeatother
\usepackage{numprint}

\begin{document}
%\listofpdfcomments
\maketitle
\begin{abstract}
{This project numerically illustrates the functions of of various state-of-the-art modules in a feed forward neural network. Section~\ref{chapter1} briefly reviews relevant work in the field of neural network, which finds neural network may be the most powerful tool for classification task without a clear mathematical structure. Section~\ref{chapter2} describes the empirical techniques applied in our neural network and their theoretical support. Our simulation results in Section~\ref{chapter4} agree with the theoretical properties of the two algorithms. Future work involves embedding the proposed backward propagation into the \textsc{nvidia} \textsc{gpu} based parallel-programming model in the \texttt{Python} package \texttt{Numba}.}
\end{abstract} 

\tableofcontents{}

\section{Introduction\label{chapter1}}
%Briefly introduce \textsc{nmf}, applications

%- What’s the aim of the study? 
% Draft Completed - Pending Review
The aim of this study is to build a multi-layer neural network application to perform a ten-class classification and use various state-of-the-art modules to improve the prediction accuracy on the given dataset.
The task provides a labelled training dataset and an unlabelled testing dataset.
The training dataset given for this task consists of $128$ features for $60,000$ samples,
the test dataset contains $10,000$ samples with same amount of features as training dataset.
Compare to a simple multi-layer neural network implementation,% @Chen review: I don't understand this sentence
our neural network includes various state-of-the-art modules to further improve the prediction accuracy and convergence rate. These methods include advanced activation functions, 
weight decay, 
momentum method,
%stochastic gradient descent methods, 
mini-batch training, batch normalisation and dropout. 
Our model successfully classifies the cross validation (CV) data set with an accuracy of $89.9\%$ in $2.3$ minutes. 
To illustrate the effects of each module, we turn the module off one by one in our benchmark model and then explore the numerical consequences.

%Literature review find a review paper read and rewrite
The multi-class neural network classification is not a trivial extension from two-class neural networks,
it requires robust, adaptive, non-parametric classifiers that can be implemented on high-speed parallel computers,
the most commonly used classifier is the hyper-plane classifiers that form complex output regions such as sigmoid or tanh,
these classifiers does not require high memory and computation complexity but they require long training time,
they include multi-layer neurons trained with back-propagation \citet{werbos1990backpropagation}.
Gradient-based back propagation is introduced to minimise the loss between actual outputs and the predicted output in the supervised learning \citet{lecun1998Gradient},
We also found multiple strategies to optimise the mini-batch gradient descent using momentum, batch normalisation and early stopping,
these strategies further improves the training accuracy and achieve fast convergence for training the neural network model \citet{ruder2016overview}.
Two common techniques are used to prevent over-fitting of the model during the experiments, dropout is a technique to randomly drop neurons from the neural network during training \citet{dropout} and weight decay can suppress some noise on the targets to improve generalisation \citet{NIPS1991_563}.

% - Why is the study important? 
% again these are about how neural network and hence moved to methods
% end to end structure
% find latent feature
% ??
% ?? refer to the mid test
Our study is important because it not only helps us to understand the recent advance of neural network,  but also validate the usefulness of these modules numerically. Neural network is widely applicable in Artificial Intelligence. It is irreplaceable by any other machine learning model because it has (num) outstanding properties.   
%Firstly, neural network has the ability to classify the data that it has never seen before. hence it can help us predict the labels for testing dataset. %who said this? I don't really understand.

Secondly, neural network can accurately classify complex and highly non-linear dataset. The mathematical structure behind Artificial Intelligence problems is usually difficult to formulated. As a result, a successful model for these complex datasets has to have a complex hypothesis class \citep{Bishop:2006:PRM:1162264}. 
With neural network, we can conveniently increase the complexity of the hypothesis class by raising either the number of layers or the number of hidden neurons per layer. 
With a suitable hypothesis class (e.g. several layers of non-linear activation functions), the neural network model is capable to extract latent features from this large amounts of data.

Thirdly, the training process of neural network can utilise modern multi-core CPUs and GPUs. The most computationally intensive part of training a neural network is to computing matrix product and Hadamard product for large matrices. Modern languages like \texttt{Python} perform these matrix operations very effectively with highly vectorised code \citet{5452452}, with either CPU or GPU.

%Lastly, neural network is fault tolerant and self-repair since it ensures reliability when some portions of the network are not working, hence we use dropout in our experiment to make the neural network more resilient. %Again, who said it, only refer to best conferences and journal papers.

%As we identify the importance of the study so we will focus on identifying the best neural network and optimisation methodologies for this purpose.



% Draft Completed - Pending Review
This report firstly introduces the theoretical supports for loss functions, various layers and optimisation techniques. 
This is followed by the experiment setup. We then compare the simulation results in a great detail.
Finally, we discuss importance of all these module and how our neural network model benefits from them.

\section{Related work}\label{chp3}
Researchers proposed many \textsc{nmf} algorithms. As the objective function of \textsc{nmf} is non-convex, for which the traditional gradient decent method could be very sensitive to step sizes and converge slowly, \citet{lee2001algorithms} first propose to algorithms which minimise Euclidean distance or Kullback-Leibler divergence~(\textsc{klnmf}) between the original matrix and its approximation. Although these algorithms are easy to implement and have reasonable convergent rate \citep{lee2001algorithms}, they require more iterations than alternatives such as gradient descent \citep{berry2007algorithms}. Also, the algorithms may fail on seriously corrupted datasets which violate its assumption of Gaussian noise or Poisson noise, respectively \citep{guan2017truncated}. Moreover, \citet{yang2011kullback} indicate that these methods are sensitive to the initial selection of matrices~$W$ and~$H$. The algorithms require many iterations to retrieve from poorly selected initial values.

Apart from different loss functions, several optimisation methods were proposed to improve the performance of \textsc{nmf}. After \citet{lee2001algorithms} proposed multiplicative update rules \textsc{mur}, \citet{ lin2007convergence} proposed a modified \textsc{mur} which guaranteed the convergence to a stationary point. This modified \textsc{mur}, however, did not improve the convergence rate of traditional \textsc{mur} \citep{guan2012nenmf}. Moreover, as \textsc{mur} does not impose sparseness, \citet{berry2007algorithms} proposed a projected nonnegative least square (\textsc{pnls}) method to enforce sparseness. In each nonnegative least square sub-problem, this algorithm projects the negative elements of least squares solution directly to zero. Nevertheless, \textsc{pnls} does not guarantee convergence \citep{guan2012nenmf}. 

In contrast to these gradient-based optimisation methods, \citet{kim2008nonnegative} presented an active set method which divides variables into two sets, a free set and an active set. They update free set in each iteration by solving an unconstrained equation. Even though the active set method has a nice convergence rate, it assumes strictly convexity in each nonnegative least square sub-problem \citep{kim2008nonnegative}. These assumptions are easily violated in real life applications.

There exist many robust \textsc{nmf} algorithms which include penalties in the objective functions. For example, \citet{lam2008non} proposes ${L_1}$-norm based \textsc{nmf} to model noisy data with a Laplace distribution which is less sensitive to outliers. However, as $L_1$-norm is not differentiable at zero, the optimisation procedure is computationally expensive. \citet{kong2011robust} proposed an \textsc{nmf} algorithm using $L_{21}$-norm loss function which is more stable. The updating rules used in $L_{21}$-norm \textsc{nmf}, however, still converge slowly because of continual use of the power method \citep{guan2017truncated}.



%!TEX root = report.tex
\section{Methods \label{chapter2}}
%\subsection{Noise Design}
Some carefully designed \textsc{nmf} are robust to various noises. These robust algorithms aim to significantly reduce the amount of noise while preserving the edges without blurring the images \citep{barbu2013variational}. %Figure~\ref{noises} shows three kinds of noises we designed, including Gaussian noise, Poisson noise, and Salt \& Pepper noise.

\subsection{NMF and Gaussian noise}
\textbf{Gaussian noise} is noise with a probability density function being normal with mean zero. \citet{lee2001algorithms} propose the first \textsc{nmf} with the objective function between image~$V$ and its \textsc{nmf} factorisation~$W$ and~$H$ being
\begin{equation}
  \left\Vert V-WH \right\Vert= \sum_{ij} \left[V_{ij}-(WH)_{ij}\right]^2.\label{eq:obnmf}
\end{equation}
To minimise this object function of least square, \citet{lee2001algorithms} prove the convergence of the multiplication update rule
\begin{equation}
H_{jk}\leftarrow H_{jk}\frac{(W^{T}V)_{jk}}{(W^{T}WH)_{jk}} \text{ and } W_{ij}\leftarrow W_{ij}\frac{(VH)_{ij}}{(WHH^{T})_{ij}}.\label{eq:nmf}
\end{equation}
Here, $()_{ij}/()_{ij}$ denotes an entry-wise division of the two matrices. \citet{liu2015performance} show this \textsc{nmf} algorithm minimises Gaussian.

\subsection{KLNMF and Poisson noise}
\textbf{Poisson noise} or shot noise is a type of electronic noise that
occurs when the finite number of particles that carry energy,
such as electrons in an electronic circuit or photons in an optical
device, is small enough to give rise to detectable statistical
fluctuations in a measurement.
\citet{lee2001algorithms} suggest that \textsc{klnmf} is an algorithm that minimising the Kullback-Leibler divergence
\begin{eqnarray}
  D(V||WH)&=&\sum_{ij}\left(V_{ij}\log\frac{V_{ij}}{\left(WH\right)_{ij}}-V_{ij}+\left(WH\right)_{ij}\right)\nonumber\\
          &=&\sum_{ij}\left(-V_{ij}\log\left(WH\right)_{ij}+\left(WH\right)_{ij}+C(V_{ij})\right).\label{eq:klobj}
\end{eqnarray}
where $C(V_{ij})=V_{ij}\log V_{ij}-V_{ij}$. $C(V_{ij})$ is a function of the observed image matrix~$V$ only.
\citet{lee2001algorithms} also suggest a multiplication update rule to find as the optimisation procedure of \textsc{klnmf}
\begin{equation}
H_{jk}\leftarrow H_{jk}\frac{\sum_{i}W_{ij}V_{ik}/(WH)_{jk}}{\sum_{i'}W_{i'j}} \text{ and } W_{ij}\leftarrow W_{ij}\frac{\sum_{k}H_{jk}V_{ik}/(WH)_{jk}}{\sum_{k'}H_{ik'}}. \label{eq:klnmf}
\end{equation}
As this original image matrix~$V$ is observed, minimising this Kullback-Leibler divergence~\eqref{eq:klobj} is equivalent to minimising
\begin{equation*}
  \sum_{ij}\left(-V_{ij}\log\left(WH\right)_{ij}+\left(WH\right)_{ij}+C(V_{ij})\right).
\end{equation*},
for arbitrary bounded function~$C(V_{ij})$. Taking exponential of the negative of this score function, the problem transforms to maximising the following likelihood function
\begin{equation*}
L(WH|V)=\prod_{ij}\left(\left(WH\right)_{ij}^{V_{ij}}e^{-\left(WH\right)_{ij}}+C(V_{ij})\right).
\end{equation*}
Choosing constant $C(V_{ij})$ to be $-\log V_{ij}!$ gives
\begin{equation*}
L(WH|V)=\prod_{ij}\left(\frac{\left(WH\right)_{ij}^{V_{ij}}e^{-\left(WH\right)_{ij}}}{V_{ij}!}\right).
\end{equation*}
Hence, the probability density function of each element of the original matrix~V is Poisson
\begin{equation*}
P(V_{ij})=\frac{\left(WH\right)_{ij}^{V_{ij}}e^{-\left(WH\right)_{ij}}}{V_{ij}!}
\end{equation*}
is a sufficient condition to yield this likelihood. Hence \textsc{klnmf} is most suitable for images with Poisson noise.

\subsection{Preprocess}
We did not preprocess the images because both of \textsc{nmf} and \textsc{klnmf} are scale sensitive, because both objective functions~\eqref{eq:obnmf} and~\eqref{eq:klobj} varies when original matrix~$V$ and result matrices~$W$ and~$H$ scale proportionally, i.e. for $\lambda$ real, $D(V||WH)\neq D(\lambda V||\lambda WH)$. To overcome this issue, we could have normalised the matrices~$W$ and~$H$ in each iteration \citep{scaless}, but it will result in an even slower computation.

\subsection{Gaussian and Poisson are asymptotic equivalent}
 We design a Gaussian noise and a Poisson noise with different magnitude.
 Poisson distribution with parameter~$\lambda$ (integer) is equivalent to the sum of $\lambda$ Poisson distributions with parameter~$1$ \citep[][p. 45]{Walck:1996cca}.
 Hence for $\lambda$ large, Central Limit Theorem implies that Poisson distribution with parameter~$\lambda$ is well approximated by $N(\lambda,\lambda)$.
 When applying Poisson noise to an image, we do not have degree of freedom to choose any parameter.
 The variance is the magnitude of the pixels. To compare the robustness of \textsc{klnmf} with \textsc{nmf} with different noise, we choose the variance of Gaussian noise to be the difference from the magnitude of the pixel, that is, $N(0,\operatorname{Var})\neq N(0,V)\approx \operatorname{Poi}(V)-V$.
 Figure~\ref{noise} visualises the similarity of Poisson distribution and Normal distribution with parameter~$V=40$. To overcome this issue, the Gaussian noise we use should have very different variance in comparison with the mean of the images. The pixel mean of the \texttt{ORL} dataset is approximately $40$, and the pixel mean of the Cropped Yale set is approximately $70$. Hence, we choose the variance of the noise to be $80^2$ so that it is the way bigger than $255$, which is the maximum value of a pixel.
\begin{figure}
  \centering
  % Requires \usepackage{graphicx}
  \includegraphics[scale=.8]{noise}\\
  \caption{Compare a Gaussian noise~$N(0,40)$ with Poisson noise $\operatorname{Poi}(40)-40$. They two distributions are asymptotically equivalent and have overlapped density functions. The Gaussian noise~$N(0,80^2)$ is very different.}\label{noise}
\end{figure}

\subsection{Salt \& Pepper noise}
Apart from Gaussian and Poisson noises, we also test our two algorithms on the commonly seen \textbf{Salt \& Pepper noise}. The noise presents itself by having dark pixels in bright regions and bright pixels in dark regions \citep{sampat2005computer}.

\subsection{Multiple initial estimates assure the algorithms stable}
As discussed in the section of related work,
The problem of nonnegative matrix factorization is not a convex problem.
Hence the results update rules~\eqref{eq:nmf} and~\eqref{eq:klnmf} coverage to maybe local minima instead of global minima, depending on the initial approximation.
Our task was to compare the robustness of the algorithm, and we do not want the instability of our algorithms to affect our comparison.
To address this issue, we implement several (i.e. $n$) initial estimates for each matrix factorisation problem.
We use the factorised matrices~$W$ and~$H$ corresponding to the least residual~\eqref{eq:obnmf} and~\eqref{eq:klobj}, for \textsc{nmf} and~\textsc{klnmf}, algorithms respectively, as the final result of factorisation.
This design of multiple starting point improves the stability of the algorithms, but it requires more computational power. To improve the computational speed, we make the number~$n$ equal to the number of cores of the \textsc{cpu}. We assign each of the $n$ initial estimates randomly with a uniform distribution. Then these each of the $n$ initial estimates is assigned to a different core of the \textsc{cpu}. This boosts the \textsc{cpu} utilisation to 100\% instantly and improved the computational speed by 70\% on the \texttt{ORL} data. The following part of our code implements this idea of parallel computing.
\begin{lstlisting}[caption=Multi-start paralle computing, label=matn1]
args = zip(repeat(V,ncpu), repeat(r,ncpu), repeat(niter[name2],ncpu), repeat(min_error[name2],ncpu))
result = pool.starmap(algo, args)
\end{lstlisting}
where \texttt{algo} is the \textsc{nmf} algorithm and \texttt{niter} is the number of iterations. We use a 16-thread Xeon high performance computer to run this algorithm.
 The algorithms run in this high performance computer so that the computing time is reasonable.  The multiple initial estimates assure the algorithms are stable.

% \begin{figure}\label{hpc}
%  \centering
%  \includegraphics[scale=.5]{hpc}
%  \caption{}
%\end{figure}



\subsection{KLNMF requires more iterations}
A residual versus the number of iteration plot (Figure~\ref{error}) shows that \textsc{klnmf} converges slower than \textsc{nmf}. In the log-log plot, the slope of the \textsc{nmf} residual plot is $2.5$ times larger than that of the \textsc{klnmf} plot for the \texttt{ORL} data. The slope estimates that the rate of convergence of \textsc{nmf} is $2.5$ faster than \textsc{klnmf}. As a result, we set the number of iterations as $500$ and $1200$ for \textsc{nmf} and \textsc{klnmf} algorithms, respectively (i.e. roughly $2.5$ more iterations).
 \begin{figure}
  \centering
  % Requires \usepackage{graphicx}
  \includegraphics[scale=.8]{Error.pdf}\\
  \caption{Residual of the objective function~\eqref{eq:obnmf} and~\eqref{eq:klobj} versus the number of iterations. \textsc{nmf} converges more than twice faster comparing with \textsc{klnmf}.}\label{error}
\end{figure}

\subsection{Evaluation metrics and their confidence intervals \label{ci}}
The assignment instruction asks us to compare the performance of \textsc{nmf} and \textsc{klnmf} by using evaluations metrics including Relative Reconstruction Errors (\textsc{rre}), Average Accuracy (\textsc{aa}), Normalized Mutual Information (\textsc{nmi}). The instruction states the formulae of these metrics. However, to systematically compare the metrics, we construct an 95\% confidence interval for any metric (e.g. \textsc{rre}) by bootstrapping percentile confidence interval. The idea of bootstrapping  is straightforward---we resample a subset of $40$ samples among the sample space of $80$ Monte-Carlo simulations and calculate the mean. We repeat this process $1000$ times. The $2.5\%$ and $97.5\%$ percentiles of the $80$ resampled means are then the bootstrapping percentile confidence interval
\begin{equation}
(\textsc{rre}^*_{2.5}, \textsc{rre}^*_{97.5}), \label{eq:boot}
\end{equation}
where $\textsc{rre}^*_{\alpha}$ is the $\alpha$ percentile of the bootstrapped distribution from our sample space with $80$ Monte-Carlo simulations.  We run $80$ simulations so that the confidence interval we construct is precise.

Bootstrapping does not require the sample space follows specific distributions. We apply this nonparametric method to construct confidence interval here because we do not know about the exact distributions of these three evaluation metrics.

\subsection{Statistical method compares the robustness of algorithms}
We implement the Kolmogorov-Smirnov test to test the hypothesis that the algorithms~\textsc{nmf} and~\textsc{klnmf} have different robustness. Again Kolmogorov-Smirnov test is distribution free, so we do not need to know the distributions of the evaluation metrics.

Let $\textsc{rre}_i$ denote the \textsc{rre} generated from our \textsc{nmf} algorithm by the $i$th Monte-Carlo simulation. Define the empirical distribution of a sample set generated by algorithm~$\alpha$, perhaps the $80$ \textsc{rre} results, as
\begin{equation}\label{epdf}
  \hat{F}_{\alpha}(x)=\frac{1}{n}\sum_{i=1}^{80}1_{\textsc{rre}_i \leq x}.
\end{equation}
The test statistic is the supremum among the differences of the empirical distribution generated using definition~\eqref{epdf} \citep{Walck:1996cca}
\begin{equation}\label{teststatistic}
D=\sup _{x}\left|F_{\alpha_2}(x)-F_{\alpha_2}(x)\right|.
\end{equation}
We compare the test statistics~$D$ with the critical value of $0.215$, which corresponds to $80$ samples and a 95\% of confidence level. We reject the null hypotheses that the two algorithms produce similar \textsc{rre} (or other evaluation metrics) with 95\% confidence level if the test statistics~$D>0.215$. The technique is extended to compare \textsc{aa} and \textsc{nmi}.
%\csvautobooktabular{{"../results/statistics".csv}}
%\subsection{Preprocessing}
%We apply global centring and local centring to preprocess the image data~\texttt{Vhat}
%\begin{lstlisting}[caption=Centring image data, label=matn1]
%n_samples = len(Vhat)
%# global centering
%Vhat = Vhat - Vhat.mean(axis=0)
%# local centering
%Vhat -= Vhat.mean(axis=1).reshape(n_samples, -1)
%Vhat -= Vhat.min()
%\end{lstlisting}


\section{Experiments}\label{chapter4}

\subsection{Dataset}
We illustrate our two \textsc{nmf} algorithms on two real-world face image datasets: \texttt{ORL} and \texttt{CroppedYaleB} (\citet{belhumeur1997eigenfaces}).
Both \texttt{ORL} and \texttt{CroppedYale} datasets contain multiple images of distinct subjects with various facial expression, lighting condition, and facial details.
Images in ORL are cropped and resized to $92 \times 112$ pixels. We further rescale it to $30 \times 37$ pixels. Similarly, we reduce the size of images in \texttt{CroppedYale} to $42 \times 48$ pixels.
For each dataset, we flatten the image matrix into a vector and append them together to get a matrix $V$ with shape $d\times n$ where integer~$d$ is the number of pixels in one image and integer~$n$ is the number of images. In each epoch, we use 90\% of data.

\subsection{Noise}
We implement three kinds of noises including Gaussian noise, Poisson noise and Salt \& Pepper noise.
\subsubsection{Gaussian Noise}\label{sec:gau}
We design the Gaussian noise by a normal distribution with a mean of $0$ and a standard deviation of $80$ (Algorithm~\ref{gau}). The \texttt{ORL} dataset has a global pixel mean of $40$ and the \texttt{CroppedYale} dataset has that of $70$. Hence the designed Gaussian noise contaminates the images significantly. We choose the standard deviation to be $80$ so that our Gaussian noise is less likely to coincident with the designed Poisson noise. To satisfy the nonnegative constant, negative value in the contaminated image is set to zero.
\begin{lstlisting}[caption= Gaussian Noise Design, label=gau]
def normal(subVhat):
    """Design a Gaussian noise."""
    V_noise = np.random.normal(0, 80, subVhat.shape) #* np.sqrt(subVhat)
    V = subVhat + V_noise
    V[V < 0] = 0
    return V, V_noise
\end{lstlisting}


\subsubsection{Poisson Noise}\label{sec:poi}
The Poisson noise is not additive and has no hyperparameters to be set. Unlike Gaussian noise, contaminated images are drawn directly from Poisson distributions with parameters set to be pixel values. Then, the Poisson noise is calculated from the difference between the contaminated image and the original image, as discussed in Section~\ref{chapter2} and demonstrated in Algorithm~\ref{poi}.
\begin{lstlisting}[caption= Poisson Noise Design, label=poi]
def possion(subVhat):
    """Design a Possion noise."""
    V = np.random.poisson(subVhat)
    V_noise = V-subVhat
    return V, V_noise
\end{lstlisting}


\subsubsection{Salt \& Pepper Noise}\label{sec:sal}
% JOYCE please add here
Salt \& Pepper noise (Algorithm~\ref{salt}) is added by drawing random integers from the discrete uniform distribution of the interval $[0, 255)$ . We find the bright places in generated image and replace pixel values in the same place of the original image with the brightest value. Similarly, we also find the dark pixels in the  images and replace pixel values in the same place of original image with the  darkest pixel value. In this case, we set the pixels whose values are greater than or equal to 230 as bright pixels and pixels whose value being less than or equal to 20 as dark pixels.
\begin{lstlisting}[caption= Salt and Pepper Noise Design, label=salt]
def salt_and_pepper(subVhat):
"""Design a salt and pepper noise where make some pixel value zeros."""
  V_noise = np.random.randint(low=0, high=255, size=subVhat.shape, dtype=int)
  V = subVhat.copy()
  V[V_noise <= 20] = 0
  V[V_noise >= 230] = 255
  return V, V_noise
\end{lstlisting}

\subsection{Experiment Setup}

We apply two algorithms (\textsc{nmf} and \textsc{klnmf}) with four categories of noises (no noise, Gaussian noise, Poisson noise and Salt \& Pepper noise), which results in eight combinations in each epoch. In each epoch, we randomly select 90\% of samples to train NMF algorithms and evaluate three metrics on reconstructed images. The training will terminate when the error reaches the minimum error, or the maximum iteration is reached. The minimum error and maximum iteration are hyperparameters which we learn from iterative experiments. Our code saves the learning errors versus the number of iterations so that we could draw the plot and observe the convergence of the learning process. We increase the number of epochs and calculate the average metrics and confidence intervals.


\subsection{Experiments Results}
\subsubsection{Two algorithms output similar reconstructed images}
Figure~\ref{noisesnmff} and~\ref{noisesklnmff} visualise the original image, designed noises, corrupted images and reconstructed images from left to right. From top to bottom, the four rows correspond to no noise, Gaussian noise discussed in Section~\ref{sec:gau}, Poisson noise discussed in Section~\ref{sec:poi} and Salt \& Paper noise discussed in Section~\ref{sec:sal}.
The first row of Figure~\ref{noisesnmff} and~\ref{noisesklnmff} show both algorithms reconstructed the original image well without artificial noise and with Poisson noise.
\begin{figure}
	\centering
	\includegraphics[scale=.9]{Result_Multiplication_Euclidean_No_Noise_Comparison}\\
	\includegraphics[scale=.9]{Result_Multiplication_Euclidean_Normal_Comparison}\\
	\includegraphics[scale=.9]{Result_Multiplication_Euclidean_Poisson_Comparison}\\
	\includegraphics[scale=.9]{Result_Multiplication_Euclidean_Salt_and_Pepper_Comparison}
	\caption{The reconstructed image by \textsc{nmf}. The original images (Column~1) are combined with noises (Column~1) including Gaussian Noise with Variance~$80$ (Row~2), Poisson Noise (Row~3), and Salt \& Pepper Noise (Row~4). The corrupted images are shown in Column~3. The reconstructed images are shown in (Column~4). The reconstruction with no noise is shown in Row~1.}\label{noisesnmff}
\end{figure}
\begin{figure}
	\centering
	\includegraphics[scale=.9]{Result_Multiplication_KL_Divergence_No_Noise_Comparison}\\
	\includegraphics[scale=.9]{Result_Multiplication_KL_Divergence_Normal_Comparison}\\
	\includegraphics[scale=.9]{Result_Multiplication_KL_Divergence_Poisson_Comparison}\\
	\includegraphics[scale=.9]{Result_Multiplication_KL_Divergence_Salt_and_Pepper_Comparison}
\caption{The reconstructed image by \textsc{klnmf}. The original images (Column~1) are combined with noises (Column~1) including Gaussian Noise with Variance~$80$ (Row~2), Poisson Noise (Row~3), and Salt \& Pepper Noise (Row~4). The corrupted images are shown in Column~3. The reconstructed images are shown in (Column~4). The reconstruction with no noise is shown in Row~1.}\label{noisesklnmff}
\end{figure}
However, when the noise is large (the second and last rows in Figure~\ref{noisesnmff} and~\ref{noisesklnmff}), the quality of reconstructed images looks marginally better than the contaminated images.
This result is consistent with \citet{guan2017truncated}, who assert that \textsc{nmf} may fail to handle extremely corrupted images when we violate the assumptions on the distributions of noise.
Moreover, the difference between images generated by \textsc{nmf} and \textsc{klnmf} are not visually significant.
Hence, we implement the statistical hypothesis test to compare of \textsc{rre}s of the two algorithms.

\subsubsection{Hypothesis test distinguishes the difference in RRE}
Substitute \href{https://raw.githubusercontent.com/JoyceXinyueWang/nmf_raw_data/master/raw_result_acc.csv}{\textsc{rre} results} (Figure~\ref{histo}) from $80$ Monte-Carlo simulations of the two algorithms into Kolmogorov-Smirnovs test~\eqref{teststatistic} gives test statistics~$D=1, 1 ,0.6625$ with no noise, Gaussian noise, and Poisson noise, for the \texttt{ORL} dataset. These three test statistics are all much greater than the critical value~$0.215$. Hence there are strong statistical evidences that the performance of \textsc{nmf} and \textsc{klnmf} are different in these three problems. For salt and Pepper noise, test statistic~$D=0.2125<0.215$, hence we fail to conclude that the two methods have different robustness against Salt and Pepper noise. Further, one tail Kolmogorov-Smirnov test concludes that \textsc{klnmf} performs better reconstructing the original image with no noise and is more robust Poisson noise. 
In contrast, \textsc{nmf} is more robust against Gaussian noise, even with only $500$ iterations ($1200$ for \textsc{klnmf}).
\begin{figure}
{\centering
\includegraphics[scale=0.8]{histo}
\caption{Histogram of \textsc{rre} results from 80 Monte-Carlo simulations. Blue bars correspond to \textsc{nmf}, and the pink bars correspond to \textsc{klnmf}. The types of noise are labelled in the plot. The visualisation agrees with our statistical analysis.}
\label{histo}}
\end{figure}

Theoretical results discussed in Section~\ref{chapter2} suggest \textsc{nmf} is more robust against Gaussian noise whereas \textsc{klnmf} is more robust against Poisson noise. Our experimental results concluded from Kolmogorov-Smirnovs hypothesis tests agree with these theoretical results. Further, as both of the algorithms are not designed for Salt and Pepper noise, they have a similar performance against it.

These results can be observed by directly reading whether the confidence intervals overlap in Table~\ref{tab:ci}. Also, the statistical results agree with the visualisation in Figure~\ref{histo},~\ref{noisesnmff} and~\ref{noisesklnmff}. These results also agree with our intuition---although the differences in the robustness of the two algorithms are small, the even smaller variances in the \textsc{rre} results make them statistically different under Poisson and Gaussian noise (Figure~\ref{histo}).

With the \texttt{CroppedYale} dataset, the test results (Table~\ref{tb:yale}) agree with those of the \texttt{ORL} data as well as theories in Section~\ref{chapter2}: 1. against Poisson noise, \textsc{klnmf} with update rule~\eqref{eq:klnmf} has a lower \textsc{rre}; 2. against Gaussian noise, \textsc{nmf} with update rule~\eqref{eq:nmf} has a lower \textsc{rre}; 3. \textsc{klnmf} has a lower \textsc{rre} for images with no artificial noise. However, using \texttt{CroppedYale} dataset, the \textsc{nmf} algorithm performs much better against Salt \& Pepper noise. This might be a result of the \textsc{nmf}'s much faster convergence rate for large dataset, that is, the performance of \textsc{klnmf} might be significantly better with more iterations when processing \texttt{CroppedYale}. We did not perform more iterations for \textsc{klnmf} due to running time constraint.
%results from hypothesis tests show that \textsc{nmf} performs uniformly better than \textsc{klnmf}, suggesting more iterations are required on \textsc{klnmf} to compare these two algorithms fairly. We fail to do so because the dataset is much larger than \texttt{ORL} and our multiplicative update rules, especially for \textsc{klnmf} converge too slow.

\subsubsection{ACC and NMI results}
In terms of \textsc{acc} and \textsc{nmi}, the differences between \textsc{nmf} and \textsc{klnmf} under Gaussian noise and Salt \& Pepper noise are trivial given their overlapped confidence intervals. Under Possion noise, however, \textsc{klnmf} is superior than \textsc{nmf} for both measurements. 


\begin{table}
\caption{Average of evaluations metrics over $80$ Monte-Carlo simulations using the \texttt{ORL} dataset. The 95\% confidence intervals are calculated using bootstrap.}
\hspace{-1cm}{\small
\label{tab:ci}\begin{tabular}{l|lll}
 \hline
\texttt{ORL} dataset & \textsc{rre} & \textsc{acc} & \textsc{nmi}\tabularnewline
 \hline
\textsc{nmf} no noise & 0.1583 (0.1581, 0.1584) & 0.7364 (0.731, 0.742) & 0.8536 (0.8506, 0.8567)\tabularnewline

\textsc{nmf} Gaussian noise & 0.2925 (0.2922, 0.2927) & 0.447 (0.4423, 0.4521) & 0.6212 (0.6176, 0.6247)\tabularnewline

\textsc{nmf} Poisson noise & 0.1611 (0.161, 0.1613) & 0.7313 (0.7262, 0.7367) & 0.8493 (0.8456, 0.8527)\tabularnewline

\textsc{nmf} Salt and Pepper noise & 0.2636 (0.2634, 0.2638) & 0.5094 (0.504, 0.5151) & 0.6721 (0.6679, 0.6764)\tabularnewline

\textsc{klnmf} no noise & 0.1729 (0.1728, 0.173) & 0.7406 (0.7352, 0.7458) & 0.8599 (0.8568, 0.8632)\tabularnewline

\textsc{klnmf} Gaussian noise & 0.2977 (0.2976, 0.2979) & 0.4538 (0.4483, 0.4595) & 0.6209 (0.6165, 0.6255)\tabularnewline

\textsc{klnmf} Poisson noise & 0.1602 (0.1601, 0.1603) & 0.7417 (0.7365, 0.7472) & 0.8573 (0.8542, 0.8602)\tabularnewline

\textsc{klnmf} Salt and Pepper noise & 0.264 (0.2638, 0.2643) & 0.5089 (0.5038, 0.5139) & 0.6734 (0.6694, 0.6779)\tabularnewline
 \hline
\end{tabular}}
\end{table}

%In terms of the \texttt{CroppedYale} dataset, results from hypothesis tests show that \textsc{nmf} performs uniformly better than \textsc{klnmf}, suggesting more iterations are required on \textsc{klnmf} to compare these two algorithms fairly. We fail to do so because the dataset is much larger than \texttt{ORL} and our multiplicative update rules, especially for \textsc{klnmf} converge too slow. Note that accuracy of \texttt{CroppedYale} is small overall. This is because we rescale the images in \texttt{CroppedYale} by factor 4 whereas we only rescale the images in  \texttt{ORL} by factor 3. Besides, there are more cropped images in  \texttt{CroppedYale} and there are more labels in  \texttt{CroppedYale}. All of them contribute to the low accuracy.

\begin{table}
\caption{Average of evaluations metrics over 10 Monte-Carlo simulations using \texttt{CroppedYale} dataset.}
\centering
\hspace{-1cm}{
\label{tb:yale}\begin{tabular}{l|lll}
 \hline
\texttt{CroppedYale} dataset & \textsc{rre} & \textsc{acc} & \textsc{nmi}\tabularnewline
 \hline
\textsc{nmf} no noise & 0.2022 & 0.2377 & 0.3204\tabularnewline

\textsc{nmf} Gaussian noise & 0.3114 & 0.2116 & 0.2777\tabularnewline

\textsc{nmf} Poisson noise & 0.2023 & 0.2442 & 0.3241\tabularnewline

\textsc{nmf} Salt and Pepper noise & 0.2748 & 0.2133 & 0.2902 \tabularnewline

\textsc{klnmf} no noise & 0.2062 & 0.2434 & 0.3191 \tabularnewline

\textsc{klnmf} Gaussian noise & 0.3106 & 0.2112 & 0.2845 \tabularnewline

\textsc{klnmf} Poisson noise & 0.2065 & 0.2377 & 0.3104\tabularnewline

\textsc{klnmf} Salt and Pepper noise & 0.3164 & 0.1634 &0.2332 \tabularnewline
 \hline
\end{tabular}}
\end{table}
\subsection{Personal Reflection}
The implementation of this project was challenging and rewarding. We overcome many problems that are not taught in class by research. For instance, we learnt to use multi-start algorithm with parallel computing to overcome the problem that ~\textsc{klnmf} always gets stuck into local optimal. 
In addition, we learnt that although one algorithm performs better than another one theoratically, it might require more running time. In real implementation, we need to consider the running time versus performance trade-off. For example, the ~\textsc{klnmf}  multi-start parallel computing has overall better performance than ~\textsc{nmf}; however, it has slow convergence rate and multi-start parallel computing will cost more than than non-multi-start ~\textsc{klnmf} algorithm, especially when training \texttt{CroppedYale} dataset .
Moreover, we critically considered what truly defines a `good' algorithm. Although theoretically privileged algorithms may be good for handling difficult tasks, they tend to be time-consuming and not easy to implement. For example, through literature review, we found some algorithms such as Truncated Cauchy \textsc{nmf} \citet{guan2017truncated} that are excellent for contaminated data but too difficult for us to implement. Hence, in real-world practice, simpler and faster algorithm may be more widely used than advanced algorithms. 
Lastly, we observed many interesting results during the experiment. For instance, we found that the \textsc{rre}s of \textsc{klnmf} with Poisson noise is superior to that with no noise in \texttt{ORL} dataset. %One hypothesis is that added Poisson noise neutralised the noise from the original image. Another hypothesis is that 
This may result from the noise assumptions made by these two algorithms. 


\section{Conclusion}
In conclusion, our numerical simulation supports the theoretical results that \textsc{nmf} based on objective function~\eqref{eq:obnmf} is more robust to Gaussian noise and \textsc{klnmf} based on objective function~\eqref{eq:klobj} is more robust to Poisson noise. The two algorithms have a similar performance against Pepper \& Salt noise. However, the \textsc{nmf} algorithm reconstructs image better without noise and converges much faster with the multiplicative update rule when comparing with \textsc{klnmf}. Also, simulation results find both of the multiplicative update rules are sensitive to the initial values of matrices~$W$ and~$H$. Section~\ref{chapter2} proposes a solution based on parallel programming to solve this problem. However, this solution requires high performance computer so that the algorithm converges in a reasonable amount of time.

Recently, \textsc{nvidia} released their \href{https://developer.nvidia.com/cuda-zone}{\texttt{Cuda}} package which parallelises algorithms using \textsc{gpu}. This package improves the speed of parallelisable algorithms, including many \textsc{nmf} algorithms, by a factor of $>100$. As a computationally expensive procedure, our suggestion of using multiple initialisations will be more novel if a \textsc{gpu} version based on \href{https://developer.nvidia.com/cuda-zone}{\texttt{Cuda}} could be designed. 


%\input{appendix}

\titleformat{\section}
    {\normalsize\bfseries\centering}{\thesection}{5pt}{\normalsize}
\bibliography{research}

\end{document}
